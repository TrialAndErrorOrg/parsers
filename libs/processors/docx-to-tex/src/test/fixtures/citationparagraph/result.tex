 \documentclass{article}
\usepackage[style=apa]{biblatex}
\addbibresource{bibliography.bib}\usepackage{graphicx}\usepackage{hyperref}



\begin{filecontents}{bibliography.bib}

  @article{Saebi2019,
    title       = {Social entrepreneurship research: Past achievements and future promises},
    author      = {Saebi, T. and Foss, N.J. and Linder, S.},
    number      = {1},
    volume      = {45},
    doi         = {10.1177/0149206318793196},
    year        = {2019},
    pages       = {70--95},
    journal     = {Journal of Management}
}


@article{Mair2010,
    title       = {Social entrepreneurship: taking stock and looking ahead},
    author      = {Mair, J.},
    publisher   = {Edward Elgar Publishing Limited},
    year        = {2010},
    pages       = {15--28},
    journal     = {Handbook of Research on Social Entrepreneurship}
}


@article{Choi2014,
    title       = {Social entrepreneurship as an essentially contested concept: Opening a new avenue for systematic future research},
    author      = {Choi, N. and Majumdar, S.},
    number      = {3},
    volume      = {29},
    doi         = {10.1016/j.jbusvent.2013.05.001},
    year        = {2014},
    pages       = {363--376},
    journal     = {Journal of Business Venturing}
}


@article{Alegre2017,
    title       = {Organized Chaos: Mapping the Definitions of Social Entrepreneurship},
    author      = {Alegre, I. and Kislenko, S. and Berbegal-Mirabent, J.},
    number      = {2},
    volume      = {8},
    doi         = {10.1080/19420676.2017.1371631},
    year        = {2017},
    pages       = {248--264},
    journal     = {Journal of Social Entrepreneurship}
}


@article{Stephan2015,
    title       = {Institutions and social entrepreneurship: The role of institutional voids, institutional support, and institutional configurations},
    author      = {Stephan, U. and Uhlaner, L.M. and Stride, C.},
    number      = {3},
    volume      = {46},
    doi         = {10.1057/jibs.2014.38},
    year        = {2015},
    pages       = {308--331},
    journal     = {Journal of International Business Studies}
}


@article{Korndörfer2015,
    title       = {A large scale test of the effect of social class on prosocial behavior},
    author      = {Korndörfer, M. and Egloff, B. and Schmukle, S.C.},
    number      = {7},
    volume      = {10},
    doi         = {10.1371/journal.pone.0133193},
    year        = {2015},
    journal     = {PloS ONE}
}


@article{Mair2006,
    title       = {Social entrepreneurship research: A source of explanation, prediction, and delight},
    author      = {Mair, J. and Martí, I.},
    number      = {1},
    volume      = {41},
    doi         = {10.1016/j.jwb.2005.09.002},
    year        = {2006},
    pages       = {36--44},
    journal     = {Journal of World Business}
}


@article{Stirzaker2021,
    title       = {The drivers of social entrepreneurship: agency, context, compassion and opportunism},
    author      = {Stirzaker, R. and Galloway, L. and Muhonen, J. and Christopoulos, D.},
    number      = {6},
    volume      = {27},
    doi         = {10.1108/IJEBR-07-2020-0461},
    year        = {2021},
    pages       = {1381--1402},
    journal     = {International Journal of Entrepreneurial Behavior \& Research}
}


@book{Inglehart1977,
    title       = {The silent revolution: Changing values and political styles in advanced industrial society},
    author      = {Inglehart, R.},
    publisher   = {Princeton University Press},
    place       = {Princeton, NJ},
    year        = {1977}
}


@article{Kraus2012,
    title       = {Social class, solipsism, and contextualism: how the rich are different from the poor},
    author      = {Kraus, M.W. and Piff, P.K. and Mendoza-Denton, R. and Rheinschmidt, M.L. and Keltner, D.},
    number      = {3},
    volume      = {119},
    doi         = {10.1037/a0028756},
    year        = {2012},
    pages       = {546--572},
    journal     = {Psychological Review}
}


@article{Franzen2013,
    title       = {Acquiescence and the Willingness to Pay for Environmental Protection: A Comparison of the ISSP, WVS, and EVS},
    author      = {Franzen, A. and Vogl, D.},
    number      = {3},
    volume      = {94},
    doi         = {10.1111/j.1540-6237.2012.00903.x},
    year        = {2013},
    pages       = {637--659},
    journal     = {Social Science Quarterly}
}


@article{Pathak2016,
    title       = {Informal institutions and their comparative influences on social and commercial entrepreneurship: The role of in-group collectivism and interpersonal Trust},
    author      = {Pathak, S. and Muralidharan, E.},
    number      = {S1},
    volume      = {54},
    doi         = {10.1111/jsbm.12289},
    year        = {2016},
    pages       = {168--188},
    journal     = {Journal of Small Business Management}
}


@article{Lepoutre2013,
    title       = {Designing a global standardized methodology for measuring social entrepreneurship activity: the Global Entrepreneurship Monitor social entrepreneurship study},
    author      = {Lepoutre, J. and Justo, R. and Terjesen, S. and Bosma, N.},
    number      = {3},
    volume      = {40},
    doi         = {10.1007/s11187-011-9398-4},
    year        = {2013},
    pages       = {693--714},
    journal     = {Small Business Economics}
}


@article{Defourny2010,
    title       = {Conceptions of Social Enterprise and Social Entrepreneurship in Europe and the United States: Convergences and Divergences},
    author      = {Defourny, J. and Nyssens, M.},
    number      = {1},
    volume      = {1},
    doi         = {10.1080/19420670903442053},
    year        = {2010},
    pages       = {32--53},
    journal     = {Journal of Social Entrepreneurship}
}


@article{Haigh2015,
    title       = {Hybrid organizations: origins, strategies, impacts, and implications},
    author      = {Haigh, N. and Walker, J. and Bacq, S. and Kickul, J.},
    number      = {3},
    volume      = {57},
    doi         = {10.1525/cmr.2015.57.3.5},
    year        = {2015},
    pages       = {5--12},
    journal     = {California Management Review}
}


@article{Weerawardena2006,
    title       = {Investigating social entrepreneurship: A multidimensional model},
    author      = {Weerawardena, J. and Mort, G.S.},
    number      = {1},
    volume      = {41},
    doi         = {10.1016/j.jwb.2005.09.001},
    year        = {2006},
    pages       = {21--35},
    journal     = {Journal of World Business}
}


@article{Battilana2014,
    title       = {Advancing research on hybrid organizing--Insights from the study of social enterprises},
    author      = {Battilana, J. and Lee, M.},
    number      = {1},
    volume      = {8},
    doi         = {10.1080/19416520.2014.893615},
    year        = {2014},
    pages       = {397--441},
    journal     = {The Academy of Management Annals}
}


@article{Bacq2016,
    title       = {Beyond the moral portrayal of social entrepreneurs: An empirical approach to who they are and what drives them},
    author      = {Bacq, S. and Hartog, C. and Hoogendoorn, B.},
    number      = {4},
    volume      = {133},
    doi         = {10.1007/s10551-014-2446-7},
    year        = {2016},
    pages       = {703--718},
    journal     = {Journal of Business Ethics}
}


@article{Zahra2014,
    title       = {On the frontiers: The implications of social entrepreneurship for international entrepreneurship},
    author      = {Zahra, S. and Newey, L. and Li, Y.},
    number      = {1},
    volume      = {38},
    doi         = {10.1111/etap.12061},
    year        = {2014},
    pages       = {137--158},
    journal     = {Entrepreneurship Theory and Practice}
}

\end{filecontents}

\begin{document}

  Despite the ongoing public and scholarly attention on social entrepreneurship, definitional debates seem far from settled \parencite{Saebi2019}. On the one hand, some scholars argue that defining social entrepreneurship is problematic because it means different things to different people and differs between contexts \parencite{Mair2010}. Hence making it a ‘fuzzy' \parencite{Choi2014} or an ‘unclear and contested' concept \parencite{Saebi2019}. On the other hand, others argue that a widespread consensus exists within the academic community on what defines social entrepreneurship, social entrepreneur and social enterprise \parencite{Alegre2017}. 

Wiw \parencite{Stephan2015}. Next to the motivations that serve the self-interest of individuals, a desire to help others by contributing to the greater good may also translate into self-employed entrepreneurship. Pro-social behaviour is often enabled by more financial and human capital \parencite{Korndörfer2015}. This enables people to start a social enterprise to benefit their local community \parencite{Stephan2015}, especially when people feel an ethical desire to contribute to society \parencite{Mair2006, Stirzaker2021}. Such interests, and individual-related immaterial goals, are evoked by relatively higher levels of financial capital \parencite{Inglehart1977, Kraus2012}. This is in line with the argument made by \parencite{Inglehart1977}, who argues that higher levels of financial capital may provide the basis for caring and pro-actively protecting the ecological environment \parencite{Franzen2013}. Furthermore, higher levels of human capital are associated with the propensity to start a \emph{social }enterprise \parencite{Pathak2016, Stephan2015}. 

Social enterprises are the tangible form of social entrepreneurship \parencite{Mair2006}. However, the population of social enterprises is not homogeneous. \parencite{Lepoutre2013} differentiate between \emph{explicit }and \emph{implicit} organizational forms of social entrepreneurship. The difference is that a social or environmental objective is part of the core mission or identity of the explicit organizational form, which is not present for the implicit social enterprises. Whether or not they receive additional financial support, the social and entrepreneurial dimensions are present among \emph{explicit }social enterprises (or hybrid enterprises) \parencite{Defourny2010}. The entrepreneurial dimension is manifested through a business logic \parencite[][at, ,least, ,to, ,some, ,extent, ,--see]{Lepoutre2013} to generate revenue by selling products and services via the market. Despite the idea that business logic is incompatible with the core ideals of social value creation, once intertwined, they create the necessary conditions for classifying social entrepreneurship \parencite{Defourny2010, Haigh2015, Weerawardena2006}. Thus, the related ideal-typical organizational form may focus on the alleviation of a particular social or environmental problem, apply a business logic, and may attract financial capital in ways consistent with either - or both - for-profit and non-profit models \parencite{Battilana2014, Lepoutre2013}. \emph{Implicit }social organizational forms include a broader spectrum of socially committed enterprises \parencite{Lepoutre2013}. Information on the relative importance of social value creation goals compared to financial value creation goals is useful to classify the degree of social entrepreneurship among these organizational forms \parencite{Bacq2016, Zahra2014}. Although not explicitly addressing a social mission statement, these organizations prioritize non-financial goals in their business operations \parencite{Lepoutre2013}. These organizations may focus on economic sustainability and on creating a positive impact on society and ecology.\emph{ }For example, such organizations may apply a ‘triple bottom line' or ‘people-planet-profit' logic. Whether explicit or implicit, social entrepreneurship manifests itself in a heterogeneous population of organizations. Furthermore, non-profits or NGOs are other examples of organizational forms that primarily aim for social or environmental value creation. Although an exclusive social value creation mission characterizes these organizations, they do not use a business logic to attain social or environmental impact. For example, these organizational forms are mostly dependent on governmental subsidies, membership fees or donations and cannot be observed as an organizational form of social entrepreneurship. Nevertheless, their funding enables those organizations to commit themselves exclusively to social impact goals. 



\section{References}



Abebe, M. A., \& Alvarado, D. (2018). Blessing in disguise? Social and institutional determinants of entrepreneurial intentions following involuntary job loss. \emph{Journal of Small Business Management, 56}(4), 555-572. doi:10.1111/jsbm.12303

Achterberg, P., Raven, J., \& van der Veen, R. (2013). Individualization: A double-edged sword: Welfare, the experience of social risks and the need for social insurance in the Netherlands. \emph{Current Sociology, 61}(7), 949-965. doi:10.1177/0011392113499738

Acs, Z. (2006). How is entrepreneurship good for economic growth? \emph{Innovations: Technology, Governance, Globalization, 1}(1), 97-107. doi:10.1162/itgg.2006.1.1.97

Agresti, A. (2018). \emph{An introduction to categorical data analysis}. John Wiley \& Sons.

Ajzen, I. (1991). The theory of planned behavior. \emph{Organizational }\emph{Behavior}\emph{ and Human Decision Processes, 50}(2), 179-211. doi:10.1016/0749-5978(91)90020-T

Akerlof, G. A. (1978). The market for “lemons”: Quality uncertainty and the market mechanism. In Diamond, P., \& Rothschild, M. (Eds.) \emph{Uncertainty in economics }(pp. 235-251). Elsevier. doi: 0.1016/B978-0-12-214850-7.50022-X

Alegre, I., Kislenko, S., \& Berbegal-Mirabent, J. (2017). Organized Chaos: Mapping the Definitions of Social Entrepreneurship. \emph{Journal of Social Entrepreneurship, 8}(2), 248-264. doi:10.1080/19420676.2017.1371631

Alter, K. (2007). \emph{Social enterprise typology}. Seattle: Virtue Ventures LLC. 

Álvarez, C., Urbano, D., \& Amorós, J. E. (2014). GEM research: achievements and challenges. \emph{Small Business Economics, 42}(3), 445-465. doi:10.1007\%252Fs11187-013-9517-5

Andersson, F. O., \& Ford, M. (2014). Reframing Social Entrepreneurship Impact: Productive, Unproductive and Destructive Outputs and Outcomes of the Milwaukee School Voucher Programme. \emph{Journal of Social Entrepreneurship, 6}(3), 299-319. doi:10.1080/19420676.2014.981845

Arvidson, M., \& Lyon, F. (2014). Social impact measurement and non-profit organisations: Compliance, resistance, and promotion. \emph{VOLUNTAS: International Journal of Voluntary and }\emph{Nonprofit}\emph{ Organizations, 25}(4), 869-886. doi:10.1007/s11266-013-9373-6

Arvidson, M., Lyon, F., McKay, S., \& Moro, D. (2013). Valuing the social? The nature and controversies of measuring social return on investment (SROI). \emph{Voluntary Sector Review, 4}(1), 3-18. doi:10.1332/204080513X661554

Audretsch, D. B., Boente, W., \& Tamvada, J. P. (2013). Religion, social class, and entrepreneurial choice. \emph{Journal of Business Venturing, 28}(6), 774-789. doi:10.1016/j.jbusvent.2013.06.002

Austin, J., Stevenson, H., \& Wei-Skillern, J. (2006). Social and commercial entrepreneurship: same, different, or both? \emph{Entrepreneurship Theory and Practice, 30}(1), 1-22. doi:10.1111/j.1540-6520.2006.00107.x

Bacq, S., Hartog, C., \& Hoogendoorn, B. (2013). A Quantitative Comparison of Social and Commercial Entrepreneurship: Toward a More Nuanced Understanding of Social Entrepreneurship Organizations in Context. \emph{Journal of Social Entrepreneurship, 4}(1), 40-68. doi:10.1080/19420676.2012.758653

Bacq, S., Hartog, C., \& Hoogendoorn, B. (2016). Beyond the moral portrayal of social entrepreneurs: An empirical approach to who they are and what drives them. \emph{Journal of Business Ethics, 133}(4), 703-718. doi:10.1007/s10551-014-2446-7

Bacq, S., \& Janssen, F. (2011). The multiple faces of social entrepreneurship: A review of definitional issues based on geographical and thematic criteria. \emph{Entrepreneurship \& Regional Development, 23}(5-6), 373-403. doi:10.1080/08985626.2011.577242

Barraket, J., \& Yousefpour, N. (2013). Evaluation and social impact measurement amongst small to medium social enterprises: Process, purpose and value. \emph{Australian Journal of Public Administration, 72}(4), 447-458. doi:10.1111/1467-8500.12042

Barton, M., Schaefer, R., \& Canavati, S. (2018). To be or not to be a social entrepreneur: Motivational drivers amongst American business students. \emph{Entrepreneurial Business and Economics Review, 6}(1), 9-35. doi:10.15678/EBER.2018.060101

Batson, C. D., Ahmad, N., \& Tsang, J. A. (2002). Four motives for community involvement. \emph{Journal of Social Issues, 58}(3), 429-445. doi:10.1111/1540-4560.00269

Battilana, J., \& Lee, M. (2014). Advancing research on hybrid organizing--Insights from the study of social enterprises. \emph{The Academy of Management Annals, 8}(1), 397-441. doi:10.1080/19416520.2014.893615

Battilana, J., Sengul, M., Pache, A.-C., \& Model, J. (2015). Harnessing productive tensions in hybrid organizations: The case of work integration social enterprises. \emph{Academy of Management Journal, 58}(6), 1658-1685. doi:10.5465/amj.2013.0903

Baum, J. A. C., \& Oliver, C. (1992). Institutional embeddedness and the dynamics of organizational populations. \emph{American Sociological Review}, 540-559. doi:10.2307/2096100

Baum, J. A. C., \& Powell, W. W. (1995). Cultivating an institutional ecology of organizations: Comment on Hannan, Carroll, Dundon, and Torres. \emph{American Sociological Review, 60}(4), 529-538. doi:10.2307/2096292

Baum, J. A. C., \& Shipilov, A. V. (2006). \emph{Ecological Approaches to Organizations}. In Clegg, S. R., Hardy, C., Lawrence, T., \& Nord, W. R. (Eds.). \emph{The SAGE Handbook of Organization Studies}. (pp.55-110). London: SAGE Publications.

Beck, U. (1992). From industrial society to the risk society: Questions of survival, social structure and ecological enlightenment. \emph{Theory, Culture \& Society, 9}(1), 97-123. doi:10.1177/026327692009001006

Becker, S., Kunze, C., \& Vancea, M. (2017). Community energy and social entrepreneurship: Addressing purpose, organisation and embeddedness of renewable energy projects. \emph{Journal of Cleaner Production, 147}, 25-36. doi:10.1016/j.jclepro.2017.01.048

Beisland, L. A., Djan, K. O., Mersland, R., \& Randøy, T. (2020). Measuring Social Performance in Social Enterprises: A Global Study of Microfinance Institutions. \emph{Journal of Business Ethics}, 171, 51-71. doi:10.1007/s10551-019-04417-z

Belmi, P., \& Laurin, K. (2016). Who wants to get to the top? Class and lay theories about power. \emph{Journal of Personality and Social Psychology, 111}(4), 505-529. doi:10.1037/pspi0000060

Benz, M., \& Frey, B. S. (2008). The value of doing what you like: Evidence from the self-employed in 23 countries. \emph{Journal of Economic }\emph{Behavior}\emph{ \& Organization, 68}(3-4), 445-455. doi:10.1016/j.jebo.2006.10.014

Blekesaune, M., \& Quadagno, J. (2003). Public Attitudes toward Welfare State PoliciesA Comparative Analysis of 24 Nations. \emph{European Sociological Review, 19}(5), 415-427. doi:10.1093/esr/19.5.415

Block, J., Thurik, R., Van der Zwan, P., \& Walter, S. (2013). Business takeover or new venture? Individual and environmental determinants from a cross--country study. \emph{Entrepreneurship Theory and Practice, 37}(5), 1099-1121. doi:10.1111/j.1540-6520.2012.00521.x

Boden Jr, R. J. (1999). Flexible working hours, family responsibilities, and female self-employment: Gender differences in self-employment selection. \emph{American Journal of Economics and Sociology, 58}(1), 71-83. doi:10.1111/j.1536-7150.1999.tb03285.x

Borzaga, C., \& Defourny, J. (2001). \emph{Conclusions. }\emph{Social enterprises in Europe: a diversity of initiatives and prospects}. In Borzaga, C., \& Defourny, J. (Eds). \emph{The Emergence of Social Enterprise. }(pp.350-370). London: Routledge.

Bosma, N. (2013). The Global Entrepreneurship Monitor (GEM) and its impact on entrepreneurship research. \emph{Foundations and Trends in Entrepreneurship, 9}(2). doi:10.1561/0300000033

Bosma, N., \& Levie, J. (2010). Global Entrepreneurship Monitor 2009 Global Report. Retrieved from https://www.gemconsortium.org/report/47107

Bosma, N., Schøtt, T., Terjesen, S. A., \& Kew, P. (2016). \emph{Global Entrepreneurship Monitor 2015 to 2016: Special topic report on social entrepreneurship}. Retrieved from https://www.gemconsortium.org/report/49542

Brady, D. (2009). \emph{Rich democracies, poor people: How politics explain poverty}. New York: Oxford University Press.

Brady, D. (2019). Theories of the Causes of Poverty. \emph{Annual Review of Sociology, 45}, 155-175. doi:10.1146/annurev-soc-073018-022550

Brieger, S. A., Bäro, A., Criaco, G., \& Terjesen, S. A. (2020). Entrepreneurs' age, institutions, and social value creation goals: A multi-country study. \emph{Small Business Economics}, \emph{57}, 425-453. doi:10.1007/s11187-020-00317-z

Brieger, S. A., \& De Clercq, D. (2019). Entrepreneurs' individual-level resources and social value creation goals: The moderating role of cultural context. \emph{International Journal of Entrepreneurial }\emph{Behavior}\emph{ \& Research, 25}(2), 193-216. doi:10.1108/IJEBR-12-2017-0503

Brieger, S. A., Terjesen, S. A., Hechavarría, D. M., \& Welzel, C. (2018). Prosociality in business: A human empowerment framework. \emph{Journal of Business Ethics}, \emph{159}, 361-380. doi:10.1007/s10551-018-4045-5

Brown, M. T., Fukunaga, C., Umemoto, D., \& Wicker, L. (1996). Annual review, 1990--1996: Social class, work, and retirement behavior. \emph{Journal of Vocational }\emph{Behavior}\emph{, 49}(2), 159-189. doi:10.1006/jvbe.1996.0039

Bryan, M. L., \& Jenkins, S. P. (2015). Multilevel modelling of country effects: A cautionary tale. \emph{European Sociological Review, 32}(1), 3-22. doi:10.1093/esr/jcv059

Bryson, J. R., \& Buttle, M. (2005). Enabling inclusion through alternative discursive formations: the regional development of community development loan funds in the United Kingdom. \emph{The Service Industries Journal, 25}(2), 273-288. doi:10.1080/0264206042000305457

Campbell, D. A. (2010). Is constituent feedback living up to its promise? Provider perceptions of feedback practices in nonprofit human service organizations. \emph{Families in Society, 91}(3), 313-320. doi:10.1606/1044-3894.4011

Campbell, D. A., \& Lambright, K. T. (2016). Program performance and multiple constituency theory. \emph{Nonprofit}\emph{ and Voluntary Sector Quarterly, 45}(1), 150-171. doi:10.1177/0899764014564578

Campbell, D. A., Lambright, K. T., \& Bronstein, L. R. (2012). In the eyes of the beholders: Feedback motivations and practices among nonprofit providers and their funders. \emph{Public Performance \& Management Review, 36}(1), 7-30. doi:10.2753/PMR1530-9576360101

Carman, J. G., \& Fredericks, K. A. (2010). Evaluation capacity and nonprofit organizations: Is the glass half-empty or half-full? \emph{American Journal of Evaluation, 31}(1), 84-104. doi:10.1177/1098214009352361

Carroll, G. R. (1985). Concentration and specialization: Dynamics of niche width in populations of organizations. \emph{American Journal of Sociology, 90}(6), 1262 -1283. doi:10.1086/228210

Carter, N. M., Gartner, W. B., Shaver, K. G., \& Gatewood, E. J. (2003). The career reasons of nascent entrepreneurs. \emph{Journal of Business Venturing, 18}(1), 13-39. doi:10.1016/S0883-9026(02)00078-2

Castellano, R., Musella, G., \& Punzo, G. (2017). Structure of the labour market and wage inequality: evidence from European countries. \emph{Quality \& Quantity, 51}(5), 2191-2218. doi:10.1007/s11135-016-0381-7

Castles, F. G. (2009). What welfare states do: a disaggregated expenditure approach. \emph{Journal of Social Policy, 38}(1), 45-62. doi:10.1017/S0047279408002547

Certo, S. T., \& Miller, T. (2008). Social entrepreneurship: Key issues and concepts. \emph{Business Horizons, 51}(4), 267-271. doi:10.1016/j.bushor.2008.02.009

Chan, A., Ryan, S., \& Quarter, J. (2017). Supported Social Enterprise: A Modified Social Welfare Organization. \emph{Nonprofit}\emph{ and Voluntary Sector Quarterly, 46}(2), 261-279. doi:10.1177/0899764016655620

Chell, E., Spence, L. J., Perrini, F., \& Harris, J. D. (2014). Social Entrepreneurship and Business Ethics: Does Social Equal Ethical? \emph{Journal of Business Ethics, 133}(4), 619-625. doi:10.1007/s10551-014-2439-6

Chmelik, E., Musteen, M., \& Ahsan, M. (2016). Measures of performance in the context of international social ventures: An exploratory study. \emph{Journal of Social Entrepreneurship, 7}(1), 74-100. doi:10.1080/19420676.2014.997781

Choi, N., \& Majumdar, S. (2014). Social entrepreneurship as an essentially contested concept: Opening a new avenue for systematic future research. \emph{Journal of Business Venturing, 29}(3), 363-376. doi:10.1016/j.jbusvent.2013.05.001

Christopoulos, D., \& Vogl, S. (2015). The motivation of social entrepreneurs: The roles, agendas and relations of altruistic economic actors. \emph{Journal of Social Entrepreneurship, 6}(1), 1-30. doi:10.1080/19420676.2014.954254

Cohen, B., \& Winn, M. I. (2007). Market imperfections, opportunity and sustainable entrepreneurship. \emph{Journal of Business Venturing, 22}(1), 29-49. doi:10.1016/j.jbusvent.2004.12.001

Coppedge, M., Gerring, J., Knutsen, C. H., Lindberg, S. I., Teorell, J., Altman, D., \& Ziblatt, D. (2021). \emph{V-Dem Country-Year Dataset v11.1}. 

Corner, P. D., \& Ho, M. (2010). How opportunities develop in social entrepreneurship. \emph{Entrepreneurship Theory and Practice, 34}(4), 635-659. doi:10.1111/j.1540-6520.2010.00382.x

Coskun, M. E., Monroe-White, T., \& Kerlin, J. (2019). An updated quantitative analysis of Kerlin's macro-institutional social enterprise framework. \emph{Social Enterprise Journal}, \emph{15}(1), 111-130. doi:10.1108/SEJ-03-2018-0032

Cowling, M., \& Bygrave, W. D. (2006). Entrepreneurship, Welfare Provision, and Unemployment: Relationships between Unemployment, Welfare Provisions, and Entrepreneurship in Thirty-Seven Nations Participating in the Global Entrepreneurship Monitor (GEM) 2002. \emph{Comparative }\emph{Labor}\emph{ Law \& Policy Journal, 28(4)}, 617-638. Retrieved from https://ssrn.com/abstract=1006267

Croson, D. C., \& Minniti, M. (2012). Slipping the surly bonds: The value of autonomy in self-employment. \emph{Journal of Economic Psychology, 33}(2), 355-365. doi:10.1016/j.joep.2011.05.001

Cutt, J., \& Murray, V. (2000). \emph{Accountability and effectiveness evaluation in non-profit organizations}. London: Routledge.Dacin, P., Dacin, M., \& Matear, M. (2010). Social entrepreneurship: Why we don't need a new theory and how we move forward from here. \emph{The Academy of Management Perspectives, 24}(3), 37-57. doi:10.5465/amp.24.3.37

Dart, R. (2004). The legitimacy of social enterprise. \emph{Nonprofit}\emph{ Management \& Leadership, 14}(4), 411-424. doi:10.1002/nml.43

Dawson, C., Henley, A., \& Latreille, P. (2014). Individual motives for choosing self-employment in the UK: Does region matter? \emph{Regional Studies, 48}(5), 804-822. doi:10.1080/00343404.2012.697140

De Clercq, D., Lim, D. S., \& Oh, C. H. (2013). Individual--level resources and new business activity: The contingent role of institutional context. \emph{Entrepreneurship Theory and Practice, 37}(2), 303-330. doi:10.1111/j.1540-6520.2011.00470.x

Dean, T. J., Brown, R. L., \& Stango, V. (2000). Environmental regulation as a barrier to the formation of small manufacturing establishments: A longitudinal examination. \emph{Journal of Environmental Economics and Management, 40}(1), 56-75. doi:10.1006/jeem.1999.1105

Dees, J. G. (1998). The meaning of “social entrepreneurship”. Retrieved from https://web.stanford.edu/class/e145/2007\_fall/materials/dees\_SE.pdf

Dees, J. G., \& Anderson, B. (2006). \emph{Framing a theory of social entrepreneurship: Building on two schools of practice and thought}. In Mosher-Williams, R. (Ed). \emph{Research on social entrepreneurship: Understanding and contributing to an emerging field}. \emph{ARNOVA Occasional Paper Series}, \emph{1}(3), 39-66. 

Defourny, J. (2001). \emph{Introduction: From Third Sector to Social Enterprise}. In Borzaga, C. and Defourny, J. (Eds) \emph{The Emergence of Social Enterprise}. (pp.1-28). London: Routledge.

Defourny, J., \& Nyssens, M. (2008). Social enterprise in Europe: recent trends and developments. \emph{Social Enterprise Journal, 4}(3), 202-228. doi:10.1108/17508610810922703

Defourny, J., \& Nyssens, M. (2010a). Conceptions of Social Enterprise and Social Entrepreneurship in Europe and the United States: Convergences and Divergences. \emph{Journal of Social Entrepreneurship, 1}(1), 32-53. doi:10.1080/19420670903442053

Defourny, J., \& Nyssens, M. (2010b). Social enterprise in Europe: At the crossroads of market, public policies and third sector. \emph{Policy and Society, 29}(3), 231-242. doi:10.1016/j.polsoc.2010.07.002

Defourny, J., \& Nyssens, M. (2017). Fundamentals for an international typology of social enterprise models. \emph{VOLUNTAS: International Journal of Voluntary and }\emph{Nonprofit}\emph{ Organizations, 28}(6), 2469-2497. doi:10.1007/s11266-017-9884-7

Dentchev, N. A. (2020). Meer sociaal ondernemerschap in tijden van COVID-19. In Brengman, M. (Ed.), \emph{Post viraal}\emph{ naar een nieuw normaal: VUB-stemmen over de impact van corona op onze samenleving} (pp.217-223). Brussel: VUBPRESS.

Dewilde, C. (2006). Becoming poor in Belgium and Britain: the impact of demographic and labour market events. \emph{Sociological Research Online, 11}(1), 87-103. doi:10.5153/sro.1206

Dickel, P., Sienknecht, M., \& Hörisch, J. (2021). The early bird catches the worm: an empirical analysis of imprinting in social entrepreneurship. \emph{Journal of Business Economics, 91}(2), 127-150. doi:10.1007/s11573-020-00969

Dileo, I., \& Pereiro, T. G. (2019). Assessing the impact of individual and context factors on the entrepreneurial process. A cross-country multilevel approach. \emph{International Entrepreneurship and Management Journal, 15}(4), 1393-1441. doi:10.1007/s11365-018-0528-1

DiMaggio, P. J., \& Powell, W. W. (1983). The iron cage revisited: Institutional isomorphism and collective rationality in organizational fields. \emph{American Sociological Review}, 147-160. doi:10.2307/2095101

Doherty, B., Haugh, H., \& Lyon, F. (2014). Social enterprises as hybrid organizations: A review and research agenda. \emph{International Journal of Management Reviews, 16}(4), 417-436. doi:10.1111/ijmr.12028

Douglas, H. (2010). Divergent orientations of social entrepreneurship organisations. In K. Hockerts, J. Mair, \& J. Robinon (Eds.), \emph{Values and opportunities in social entrepreneurship} (pp.71-90). New York: Palgrave Macmillan.

Dvouletý, O. (2018). Determinants of self-employment with and without employees: Empirical findings from Europe. \emph{International Review of Entrepreneurship, 16}(3). 

Ebrahim, A. (2003). Accountability in practice: Mechanisms for NGOs. \emph{World Development, 31}(5), 813-829. doi:10.1016/S0305-750X(03)00014-7

Ebrahim, A. (2005). Accountability myopia: Losing sight of organizational learning. \emph{Nonprofit}\emph{ and Voluntary Sector Quarterly, 34}(1), 56-87. doi: 10.1177/0899764004269430

Ebrahim, A., Battilana, J., \& Mair, J. (2014). The governance of social enterprises: Mission drift and accountability challenges in hybrid organizations. \emph{Research in Organizational }\emph{Behavior}\emph{, 34}, 81-100. doi:10.1016/j.riob.2014.09.001

Ebrahim, A., \& Rangan, V. K. (2014). What impact? A framework for measuring the scale and scope of social performance. \emph{California Management Review, 56}(3), 118-141. doi:10.1525/cmr.2014.56.3.118

Emerson, J. (2003). The blended value proposition: Integrating social and financial returns. \emph{California Management Review, 45}(4), 35-51. doi:10.2307/41166187

Erdiaw-Kwasie, M. O., Alam, K., \& Shahiduzzaman, M. (2017). Towards understanding stakeholder salience transition and relational approach to ‘better' corporate social responsibility: A case for a proposed model in practice. \emph{Journal of Business Ethics, 144}(1), 85-101. doi:10.1007/s10551-015-2805-z

Erikson, R., \& Goldthorpe, J. H. (1992). \emph{The constant flux: A study of class mobility in industrial societies.} Oxford University Press.

Esping-Andersen, G. (1990a). 4 The three political economies of the welfare state. \emph{International Journal of Sociology, 20}(3), 92-123. doi:10.1080/15579336.1990.11770001

Esping-Andersen, G. (1990b). The three worlds of welfare capitalism. Princeton: Princeton University Press.

Esping-Andersen, G. (1999). \emph{Social foundations of }\emph{postindustrial}\emph{ economies.} Oxford University Press.

Estrin, S., Mickiewicz, T., \& Stephan, U. (2013). Entrepreneurship, social capital, and institutions: Social and commercial entrepreneurship across nations. \emph{Entrepreneurship Theory and Practice, 37}(3), 479-504. doi:10.1111/etap.12019

European Commission. (2011). \emph{Flash Eurobarometer 283 (Entrepreneurship in the EU and beyond)}. \emph{GESIS Data Archive, Cologne. }\emph{ZA5439 Data file Version 1.0.0, }doi:10.4232/1.10210

European Commission. (2013). \emph{Flash Eurobarometer 354 (Entrepreneurship in the EU and beyond)}. \emph{GESIS Data Archive, Cologne. }\emph{ZA5789 Data file Version 1.0.0, }doi:10.4232/1.11590

EVS. (2016). \emph{European Values Study 2008: Integrated Dataset (EVS 2008). }\emph{GESIS Data Archive, Cologne. }\emph{ZA4800 Data file Version 4.0.0}\emph{, }. 

Fagan, E. J., Jones, B. D., \& Wlezien, C. (2017). Representative systems and policy punctuations. \emph{Journal of European Public Policy, 24}(6), 809-831. doi:10.1080/13501763.2017.1296483

Fairlie, R. W., \& Fossen, F. M. (2020). \emph{Defining opportunity versus necessity entrepreneurship: two components of business creation}. Emerald Publishing Limited.

Fauchart, E., \& Gruber, M. (2011). Darwinians, communitarians, and missionaries: The role of founder identity in entrepreneurship. \emph{Academy of Management Journal, 54}(5), 935-957. doi:10.5465/amj.2009.0211

Fitzgerald, T., \& Shepherd, D. (2018). Emerging structures for social enterprises within nonprofits: An institutional logics perspective. \emph{Nonprofit}\emph{ and Voluntary Sector Quarterly, 47}(3), 474-492. doi:10.1177/0899764018757024

Folmer, E., Rebmann, A. S., \& Stephan, U. (2016). The welfare state and social entrepreneurship: insights from a multi-level study of European regions. \emph{Frontiers of Entrepreneurship Research, 36}(15). Retrieved from https://digitalknowledge.babson.edu/fer/vol36/iss15/1

Fowler, E. A., Coffey, B. S., \& Dixon-Fowler, H. R. (2017). Transforming good intentions into social impact: A case on the creation and evolution of a social enterprise. \emph{Journal of Business Ethics, 159}(3), 665-678. doi:10.1007/s10551-017-3754-5

Franzen, A. (2003). Environmental attitudes in international comparison: An analysis of the ISSP surveys 1993 and 2000. \emph{Social Science Quarterly, 84}(2), 297-308. doi:10.1111/1540-6237.8402005

Franzen, A., \& Vogl, D. (2013). Acquiescence and the Willingness to Pay for Environmental Protection: A Comparison of the ISSP, WVS, and EVS. \emph{Social Science Quarterly, 94}(3), 637-659. doi:10.1111/j.1540-6237.2012.00903.x

Freeman, R. E. (2010). \emph{Strategic management: A stakeholder approach}. New York: Cambridge University Press.

Fukuyama, F. (2001). Social capital, civil society and development. \emph{Third World Quarterly, 22}(1), 7-20. doi:10.1080/713701144

Gelissen, J. P., Van Oorschot, W. J., \& Finsveen, E. (2012). HOW DOES THE WELFARE STATE INFLUENCE INDIVIDUALS' SOCIAL CAPITAL? Eurobarometer evidence on individuals' access to informal help. \emph{European Societies, 14}(3), 416-440. doi:10.1080/14616696.2012.676660

Germak, A. J., \& Robinson, J. A. (2014). Exploring the motivation of nascent social entrepreneurs. \emph{Journal of Social Entrepreneurship, 5}(1), 5-21. doi:10.1080/19420676.2013.820781

Giddens, A. (1998). \emph{The third way}. Cambridge: Polity Press.

Gidron, B., Kramer, R., \& Salamon, L. (1992). \emph{Government and the third sector in comparative perspective: Experience in modern welfare states}. San Francisco: Jossey-Bass.

Gidron, B., \& Monnickendam-Givon, Y. (2017). A social welfare perspective of market-oriented social enterprises. \emph{International Journal of Social Welfare, 26}(2), 127-140. doi:10.1111/ijsw.12232

Gilbert, N. (2002). \emph{Transformation of the welfare state: The silent surrender of public responsibility.} Oxford University Press.

Glänzel, G., \& Scheuerle, T. (2015). Social Impact Investing in Germany: Current Impediments from Investors' and Social Entrepreneurs' Perspectives. \emph{VOLUNTAS: International Journal of Voluntary and }\emph{Nonprofit}\emph{ Organizations, 27}(4), 1638-1668. doi:10.1007/s11266-015-9621-z

Goldthorpe, J. H., \& McKnight, A. (2006). \emph{The economic basis of social class}. Stanford University Press.

Goodin, R. E., Headey, B., Muffels, R., \& Dirven, H.-J. (1999). \emph{The real worlds of welfare capitalism}. Cambridge University Press.

Granovetter, M. (1985). Economic action and social structure: The problem of embeddedness. \emph{American Journal of Sociology, 91}(3), 481-510. doi:10.1086/228311

Gras, D., Moss, T. W., \& Lumpkin, G. T. (2014). The Use of Secondary Data in Social Entrepreneurship Research: Assessing the Field and Identifying Future Opportunities. In Short, J., Ketchen Jr., D. J., \& Bergh, D. D. \emph{Social Entrepreneurship and Research Methods.} (pp.49-75). Emerald Group Publishing Limited.

Greve, A., \& Salaff, J. W. (2003). Social networks and entrepreneurship. \emph{Entrepreneurship Theory and Practice, 28}(1), 1-22. doi:10.1111/1540-8520.00029

Greve, B. (2008). What is welfare? \emph{Central European Journal of Public Policy, 2}(1), 50-73. 

Greve, B. (2018a). Future of the welfare state? In Grave, B. (Ed). \emph{The Routledge Handbook of the Welfare State.} (pp.525-533). London: Routledge.

Greve, B. (2018b). What is welfare and public welfare? In Greve, B. (Ed.). \emph{The Routledge Handbook of the Welfare State.} (pp.512). London: Routledge.

Grieco, C. (2015). \emph{Assessing social impact of social enterprises: Does one size really fit all?} Springer.

Grieco, C. (2018). What do social entrepreneurs need to walk their talk? Understanding the attitude--behavior gap in social impact assessment practice. \emph{Nonprofit}\emph{ Management \& Leadership, 29}(1), 105-122. doi:10.1002/nml.21310

Griffiths, M. D., Henry, C., Gundry, L. K., \& Kickul, J. R. (2013). The socio-political, economic, and cultural determinants of social entrepreneurship activity. \emph{Journal of Small Business and Enterprise Development, 20}(2), 341-357. doi:10.1108/14626001311326761

Grimes, M. (2010). Strategic sensemaking within funding relationships: The effects of performance measurement on organizational identity in the social sector. \emph{Entrepreneurship Theory and Practice, 34}(4), 763-783. doi:10.1111/j.1540-6520.2010.00398.x

Haigh, N., Walker, J., Bacq, S., \& Kickul, J. (2015). Hybrid organizations: origins, strategies, impacts, and implications. \emph{California Management Review, 57}(3), 5-12. doi:10.1525/cmr.2015.57.3.5

Hannan, M. T., \& Freeman, J. (1977). The population ecology of organizations. \emph{American Journal of Sociology, 82}(5), 929-964. doi:10.1086/226424

Hannan, M. T., \& Freeman, J. (1989). \emph{Organizational ecology}. Cambridge, MA: Harvard University Press.

Hayhurst, L. M. C. (2013). The ‘Girl Effect' and martial arts: social entrepreneurship and sport, gender and development in Uganda. \emph{Gender, Place \& Culture, 21}(3), 297-315. doi:10.1080/0966369x.2013.802674

Hechavarría, D. M. (2016). The impact of culture on national prevalence rates of social and commercial entrepreneurship. \emph{International Entrepreneurship and Management Journal, 12}(4), 1025-1052. doi:10.1007/s11365-015-0376-1

Hechavarría, D. M., Terjesen, S. A., Ingram, A. E., Renko, M., Justo, R., \& Elam, A. (2017). Taking care of business: the impact of culture and gender on entrepreneurs' blended value creation goals. \emph{Small Business Economics, 48}(1), 225-257. doi:10.1007/s11187-016-9747-4

Heck, R. H., Tabata, L., \& Thomas, S. L. (2013). \emph{Multilevel and longitudinal }\emph{modeling}\emph{ with IBM SPSS.} Routledge.

Heck, R. H., Thomas, S., \& Tabata, L. (2013). \emph{Multilevel }\emph{modeling}\emph{ of categorical outcomes using IBM SPSS.} Routledge.

Henley, A. (2004). Self-employment status: The role of state dependence and initial circumstances. \emph{Small Business Economics, 22}(1), 67-82. doi:10.1023/B:SBEJ.0000011573.84746.04

Henrekson, M. (2005). Entrepreneurship: a weak link in the welfare state? \emph{Industrial and Corporate Change, 14}(3), 437-467. doi:10.1093/icc/dth060

Henrekson, M., \& Roine, J. (2006). \emph{Promoting entrepreneurship in the welfare state.} In Audretsch, D. B., Grilo, I., \& Thurik, R. (Eds.). \emph{Handbook of Research on Entrepreneurship Policy}. (pp.64-93). Edward Elgar Publishing Limited.

Henrekson, M., \& Stenkula, M. (2010). Entrepreneurship and public policy. In Acs, Z. J., \& Audretsch, D. B. (Eds). \emph{Handbook of Entrepreneurship Research} (pp.595-637). New York: Springer.

Hessels, J., Arampatzi, E., van der Zwan, P., \& Burger, M. (2018). Life satisfaction and self-employment in different types of occupations. \emph{Applied}\emph{ }\emph{Economics}\emph{ Letters, 25}(11), 734-740. doi:10.1080/13504851.2017.1361003

Hessels, J., van Gelderen, M., \& Thurik, R. (2008). Drivers of entrepreneurial aspirations at the country level: the role of start-up motivations and social security. \emph{International Entrepreneurship and Management Journal, 4}(4), 401-417. doi:10.1007/s11365-008-0083-2 

Hillman, J., Axon, S., \& Morrissey, J. (2018). Social enterprise as a potential niche innovation breakout for low carbon transition. \emph{Energy Policy, 117}, 445-456. doi:10.1016/j.enpol.2018.03.038

Hockerts, K. (2015). How hybrid organizations turn antagonistic assets into complementarities. \emph{California Management Review, 57}(3), 83-106. doi:10.1525/cmr.2015.57.3.83

Hockerts, K. (2018). The effect of experiential social entrepreneurship education on intention formation in students. \emph{Journal of Social Entrepreneurship, 9}(3), 234-256. doi:10.1080/19420676.2018.1498377

Hoogendoorn, B. (2016). The Prevalence and Determinants of Social Entrepreneurship at the Macro Level. \emph{Journal of Small Business Management, 54(S1)}, 278-296. doi:10.1111/jsbm.12301

Horemans, J., \& Marx, I. (2017). \emph{Poverty and material deprivation among the self-employed in Europe: An exploration of a relatively uncharted landscape.} Bonn: IZA - Institute of Labor Economics. Retrieved from https://ssrn.com/abstract=3041803 

Hörisch, J., Kollat, J., \& Brieger, S. A. (2017). What influences environmental entrepreneurship? A multilevel analysis of the determinants of entrepreneurs' environmental orientation. \emph{Small Business Economics, 48}(1), 47-69. doi:10.1007/s11187-016-9765-2

Hox, J. (2002). \emph{Multilevel analysis: Techniques and applications. }Mahwah, NJ: Lawrence Erlbaum Associates Publishers.

Hox, J. (2010). \emph{Multilevel analysis: Techniques and applications. }New York: Routledge.

Hughes, K. D. (2003). Pushed or pulled? Women's entry into self-employment and small business ownership. \emph{Gender, Work \& Organization, 10}(4), 433-454. doi:10.1111/1468-0432.00205

Hummels, G. (2018). The 18th Sustainable Development Goal: Social entrepreneurship in a global society. \emph{USE Working Paper series, 18}(01). 

Huysentruyt, M., Mair, J., Le Coq, C., Rimac, T., \& Stephan, U. (2016). Cross-country report: a first cross-country analysis and profiling of social enterprises prepared by the SEFORÏS research consortium. Retrieved from https://kclpure.kcl.ac.uk/portal/files/102015287/Cross\_country\_report\_6.pdf

Ilmakunnas, P., \& Kanniainen, V. (2001). Entrepreneurship, Economic Risks, and Risk Insurance in the Welfare State: Results with OECD Data 1978±93. \emph{German Economic Review, 2}(3), 195-218. doi:10.1111/1468-0475.00034

Inglehart, R. (1977). \emph{The silent revolution: Changing values and political styles in advanced industrial society}. Princeton, NJ: Princeton University Press.

Inglehart, R. (1981). Post-materialism in an environment of insecurity. \emph{The American Political Science Review}, \emph{75}(4), 880-900. doi:10.2307/1962290

Inglehart, R. (1995). Public support for environmental protection: Objective problems and subjective values in 43 societies. \emph{PS: Political Science \& Politics, 28}(1), 57-72. doi:10.2307/420583

Inglehart, R. (1997). \emph{Modernization and }\emph{postmodernization}\emph{: Cultural, economic, and political change in 43 societies}. Princeton university press.

IPCC. (2021). \emph{Climate Change 2021: The Physical Science Basis. }Retrieved from https://www.ipcc.ch/report/ar6/wg1/downloads/report/IPCC\_AR6\_WGI\_Full\_Report.pdf

Islam, A. (2015). Entrepreneurship and the allocation of government spending under imperfect markets. \emph{World Development, 70}, 108-121. doi:10.1016/j.worlddev.2015.01.002

Iversen, T. (2005). \emph{Capitalism, democracy, and welfare.} Cambridge University Press.

Jack, S. L. (2005). The role, use and activation of strong and weak network ties: A qualitative analysis. \emph{Journal of Management Studies, 42}(6), 1233-1259. doi:10.1111/j.1467-6486.2005.00540.x

Kautonen, T., Van Gelderen, M., \& Fink, M. (2015). Robustness of the theory of planned behavior in predicting entrepreneurial intentions and actions. \emph{Entrepreneurship Theory and Practice, 39}(3), 655-674. doi:10.1111/etap.12056

Kerlin, J. A. (2006). Social enterprise in the United States and Europe: Understanding and learning from the differences. \emph{VOLUNTAS: International Journal of Voluntary and }\emph{Nonprofit}\emph{ Organizations, 17}(3), 246. doi:10.1007/s11266-006-9016-2

Kerlin, J. A. (2009). \emph{Social enterprise: A global comparison}. London: University Press of New England

Kerlin, J. A. (2013). Defining social enterprise across different contexts: A conceptual framework based on institutional factors. \emph{Nonprofit}\emph{ and Voluntary Sector Quarterly, 42}(1), 84-108. doi:10.1177/0899764011433040

Kerlin, J. A. (2017). The Macro-Institutional Social Enterprise Framework: Introduction and Theoretical Underpinnings. In Kerlin, J.A. (Ed.). \emph{Shaping Social }\emph{Enterprise: Understanding Institutional Context and Influence. }(pp.1-26). Emerald Publishing Limited.

Kerlin, J. A., Monroe-White, T., \& Zook, S. (2016). \emph{Habitats in the zoo}. In Young, D. R., Searing, E. A. M., \& Brewer, C. V. (Eds.). (pp67-92). Edward Elgar Publishing.

Kibler, E., Salmivaara, V., Stenholm, P., \& Terjesen, S. (2018). The evaluative legitimacy of social entrepreneurship in capitalist welfare systems. \emph{Journal of World Business, 53}(6), 944-957. doi:10.1016/j.jwb.2018.08.002

Kickul, J., Gundry, L., Mitra, P., \& Berçot, L. (2018). Designing with purpose: advocating innovation, impact, sustainability, and scale in social entrepreneurship education. \emph{Entrepreneurship Education and Pedagogy, 1}(2), 205-221. doi:10.1177/2515127418772177

Koellinger, P., \& Minniti, M. (2009). Unemployment benefits crowd out nascent entrepreneurial activity. \emph{Economics letters, 103}(2), 96-98. doi:10.1016/j.econlet.2009.02.002

Kolvereid, L. (1996). Organizational employment versus self-employment: Reasons for career choice intentions. \emph{Entrepreneurship Theory and Practice, 20}(3), 23-31. doi:10.1177/104225879602000302

Kolvereid, L. (2016). Preference for self-employment: Prediction of new business start-up intentions and efforts. \emph{The International Journal of Entrepreneurship and Innovation, 17}(2), 100-109. doi:10.1177/1465750316648576

Korndörfer, M., Egloff, B., \& Schmukle, S. C. (2015). A large scale test of the effect of social class on prosocial behavior. \emph{PloS}\emph{ ONE, 10}(7). doi:10.1371/journal.pone.0133193

Korpi, W., \& Palme, J. (2003). New Politics and Class Politics in the Context of Austerity and Globalization: Welfare State Regress in 18 Countries, 1975--95. \emph{American Political Science Review, 97}(03). doi:10.1017/s0003055403000789

Kraus, M. W., Piff, P. K., Mendoza-Denton, R., Rheinschmidt, M. L., \& Keltner, D. (2012). Social class, solipsism, and contextualism: how the rich are different from the poor. \emph{Psychological Review, 119}(3), 546-572. doi:10.1037/a0028756

Krueger, N., Reilly, M. D., \& Carsrud, A. L. (2000). Competing models of entrepreneurial intentions. \emph{Journal of Business Venturing, 15}(5-6), 411-432. doi:10.1016/S0883-9026(98)00033-0

Kulin, J., \& Meuleman, B. (2015). Human values and welfare state support in Europe: An east--west divide? \emph{European Sociological Review, 31}(4), 418-432. doi:10.1093/esr/jcv001

Lall, S. (2017). Measuring to improve versus measuring to prove: Understanding the adoption of social performance measurement practices in nascent social enterprises. \emph{VOLUNTAS: International Journal of Voluntary and }\emph{Nonprofit}\emph{ Organizations, 28}(6), 2633-2657. doi:10.1007/s11266-017-9898-1

Lall, S. (2019). From Legitimacy to Learning: How Impact Measurement Perceptions and Practices Evolve in Social Enterprise--Social Finance Organization Relationships. \emph{VOLUNTAS: International Journal of Voluntary and }\emph{Nonprofit}\emph{ Organizations, 30}(3), 562-577. doi:10.1007/s11266-018-00081-5

Laville, J-L., Lemaître, A., \& Nyssens, M. (2006). \emph{Public policies and social enterprises in Europe: the challenge of institutionalization}. In Nyssens, M. (Ed.) \emph{Social Enterprise: At the crossroads of market, public policies and civil society}. (pp.272-295). London: Routledge.

Lepoutre, J., Justo, R., Terjesen, S., \& Bosma, N. (2013). Designing a global standardized methodology for measuring social entrepreneurship activity: the Global Entrepreneurship Monitor social entrepreneurship study. \emph{Small Business Economics, 40}(3), 693-714. doi:10.1007/s11187-011-9398-4

Liket, K., \& Maas, K. (2016). Strategic philanthropy: Corporate measurement of philanthropic impacts as a requirement for a “happy marriage” of business and society. \emph{Business \& Society, 55}(6), 889-921. doi:10.1177/0007650314565356

Liket, K. C., \& Maas, K. (2015). Nonprofit organizational effectiveness: Analysis of best practices. \emph{Nonprofit}\emph{ and Voluntary Sector Quarterly, 44}(2), 268-296. doi:10.1177/0899764013510064

Lindbeck, A. (1994). The welfare state and the employment problem. \emph{The American Economic Review, 84}(2), 71-75. 

Lindbeck, A., \& Nyberg, S. (2006). Raising children to work hard: altruism, work norms, and social insurance. \emph{The Quarterly Journal of Economics, 121}(4), 1473-1503. doi:10.1093/qje/121.4.1473

Littlewood, D., \& Holt, D. (2018). Social entrepreneurship in South Africa: Exploring the influence of environment. \emph{Business \& Society, 57}(3), 525-561. doi:10.1177/0007650315613293

Maas, C. J., \& Hox, J. J. (2005). Sufficient sample sizes for multilevel modeling. \emph{Methodology: European Journal of Research Methods for the }\emph{Behavioral}\emph{ and Social Sciences, 1}(3), 86-92. \emph{doi:10.1027/1614-2241.1.3.86}

Maas, K., \& Boons, F. (2010). CSR as a strategic activity: Value creation, redistribution and integration. In Louche, C., Idowu, S. O., \& Filho, W. L. (Eds.). \emph{Innovative CSR: From risk management to value creation. }(pp.154-172). London: Routledge. 

Maas, K., \& Grieco, C. (2017). Distinguishing game changers from boastful charlatans: Which social enterprises measure their impact? \emph{Journal of Social Entrepreneurship, 8}(1), 110-128. doi:10.1080/19420676.2017.1304435

Maas, K., \& Liket, K. (2011). Talk the Walk: Measuring the impact of strategic philanthropy. \emph{Journal of Business Ethics, 100}, 445-464. doi:10.1007/s10551-010-0690-z

Mair, J. (2010). \emph{Social entrepreneurship: taking stock and looking ahead}. In Fayolle, A., \& Matlay, H. (Eds.). \emph{Handbook of Research on Social Entrepreneurship}. (pp.15-28). Edward Elgar Publishing Limited.

Mair, J., Battilana, J., \& Cardenas, J. (2012). Organizing for society: A typology of social entrepreneuring models. \emph{Journal of Business Ethics, 111}(3), 353-373. doi:10.1007/s10551-012-1414-3

Mair, J., \& Marti, I. (2009). Entrepreneurship in and around institutional voids: A case study from Bangladesh. \emph{Journal of Business Venturing, 24}(5), 419-435. doi:10.1016/j.jbusvent.2008.04.006

Mair, J., \& Martí, I. (2006). Social entrepreneurship research: A source of explanation, prediction, and delight. \emph{Journal of World Business, 41}(1), 36-44. doi:10.1016/j.jwb.2005.09.002

Mair, J., Martí, I., \& Ventresca, M. J. (2012). Building inclusive markets in rural Bangladesh: How intermediaries work institutional voids. \emph{Academy of Management Journal, 55}(4), 819-850. doi:10.5465/amj.2010.0627

Mair, J., \& Noboa, E. (2006). \emph{Social entrepreneurship: How intentions to create a social venture are formed}. In Mair, J., Robinson, J. , \& Hockerts, K. (Eds.). \emph{Social entrepreneurship.} (pp. 121-135). Palgrave MacMillan.

Mascini, P., Achterberg, P., \& Houtman, D. (2013). Neoliberalism and work-related risks: Individual or collective responsibilization? \emph{Journal of Risk Research, 16}(10), 1209-1224. doi:10.1080/13669877.2012.761274

Maslow, A. H. (1970). New introduction: Religions, values, and peak-experiences. \emph{Journal of Transpersonal Psychology, 2}(2), 83-90. 

Mc Intyre, K., Lanting, P., Deelen, P., Wiersma, H. H., Vonk, J. M., Ori, A. P., . . . Boulogne, F. (2021). Lifelines COVID-19 cohort: investigating COVID-19 infection and its health and societal impacts in a Dutch population-based cohort. \emph{BMJ Open, 11}(3), 1-22. doi:10.1136/bmjopen-2020-044474

McMullen, J. S. (2011). Delineating the domain of development entrepreneurship: A market--based approach to facilitating inclusive economic growth. \emph{Entrepreneurship Theory and Practice, 35}(1), 185-215. doi:10.1111/j.1540-6520.2010.00428.x

Meek, W. R., Pacheco, D. F., \& York, J. G. (2010). The impact of social norms on entrepreneurial action: Evidence from the environmental entrepreneurship context. \emph{Journal of Business Venturing, 25}(5), 493-509. doi:10.1016/j.jbusvent.2009.09.007

Meyer, J. W., \& Rowan, B. (1977). Institutionalized organizations: Formal structure as myth and ceremony. \emph{American Journal of Sociology, 83}(2), 340-363. doi:10.1086/226550

Millán, J. M., Hessels, J., Thurik, R., \& Aguado, R. (2013). Determinants of job satisfaction: a European comparison of self-employed and paid employees. \emph{Small Business Economics, 40}(3), 651-670. doi:10.1007/s11187-011-9380-1

Monge, C. B. (2018). Identifying Cross-Country Key Drivers of Social Entrepreneurial Activity. \emph{Journal of Social Entrepreneurship, 9}(3), 181-199. doi:10.1080/19420676.2018.1467333

Monroe-White, T., Kerlin, J. A., \& Zook, S. (2015). A quantitative critique of Kerlin's macro-institutional social enterprise framework. \emph{Social Enterprise Journal}, \emph{11}(2), 178-201. doi:10.1108/SEJ-03-2015-0008

Moore, C. S., \& Mueller, R. E. (2002). The transition from paid to self-employment in Canada: the importance of push factors. \emph{Applied Economics, 34}(6), 791-801. doi:10.1080/00036840110058473

Moran, P. (2005). Structural vs. relational embeddedness: social capital and managerial performance. \emph{Strategic Management Journal}, \emph{26}(12), 1129-1151. doi: 10.1002/smj.486

Moulick, A. G., Alexiou, K., Kennedy, E. D., \& Parris, D. L. (2020). A total eclipse of the heart: compensation strategies in entrepreneurial nonprofits. \emph{Journal of Business Venturing, 35}(4), 105950. doi:10.1016/j.jbusvent.2019.105950

Mühlböck, M., Warmuth, J.-R., Holienka, M., \& Kittel, B. (2018). Desperate entrepreneurs: no opportunities, no skills. \emph{International Entrepreneurship and Management Journal, 14}(4), 975-997. doi:10.1007/s11365-017-0472-5

Muuri, A. (2010). The impact of the use of the social welfare services or social security benefits on attitudes to social welfare policies. \emph{International Journal of Social Welfare, 19}(2), 182-193. doi:10.1111/j.1468-2397.2009.00641.x

Nahapiet, J., \& Ghoshal, S. (1998). Social capital, intellectual capital and the organizational advantage. \emph{Academy of Management Review}, \emph{38}(2), 242-266. doi: 10.5465/amr.1998.533225

Newcomer, K., Baradei, L. E., \& Garcia, S. (2013). Expectations and capacity of performance measurement in NGOs in the development context. \emph{Public Administration and Development, 33}(1), 62-79. doi:10.1002/pad.1633

Nga, J. K. H., \& Shamuganathan, G. (2010). The influence of personality traits and demographic factors on social entrepreneurship start up intentions. \emph{Journal of Business Ethics, 95}(2), 259-282. doi:10.1007/s10551-009-0358-8

Nguyen, L., Szkudlarek, B., \& Seymour, R. G. (2015). Social impact measurement in social enterprises: An interdependence perspective. \emph{Canadian Journal of Administrative Sciences/Revue Canadienne des Sciences de }\emph{l'Administration}\emph{, 32}(4), 224-237. doi:10.1002/cjas.1359

Nicholls, A. (2009). ‘We do good things, don't we?': 'Blended Value Accounting' in social entrepreneurship. \emph{Accounting, Organizations and Society, 34}(6-7), 755-769. doi:10.1016/j.aos.2009.04.008

Nicholls, A. (2010a). The institutionalization of social investment: The interplay of investment logics and investor rationalities. \emph{Journal of Social Entrepreneurship, 1}(1), 70-100. doi:10.1080/19420671003701257

Nicholls, A. (2010b). The legitimacy of social entrepreneurship: Reflexive isomorphism in a pre--paradigmatic field. \emph{Entrepreneurship Theory and Practice, 34}(4), 611-633. doi:10.1111/j.1540-6520.2010.00397.x

Nicholls, A., \& Cho, A. (2006). \emph{Social Entrepreneurship: The Structuration of a field}. In Nicholls, A. (Ed.). \emph{Social entrepreneurship: }\emph{New}\emph{ paradigms of sustainable social change.} (pp.99-118). Oxford: Oxford University Press.

Nyssens, M. (2006). \emph{Social enterprise: At the crossroads of market, public policies and} \emph{civil} \emph{society}. Routledge. 

Oesch, D. (2008). The changing shape of class voting: An individual-level analysis of party support in Britain, Germany and Switzerland. \emph{European Societies, 10}(3), 329-355. doi:10.1080/14616690701846946

Ormiston, J. (2019). Blending practice worlds: Impact assessment as a transdisciplinary practice. \emph{Business Ethics: A European Review, 28}(4), 423-440. doi:10.1111/beer.12230 

Ormiston, J., Charlton, K., Donald, M. S., \& Seymour, R. G. (2015). Overcoming the challenges of impact investing: Insights from leading investors. \emph{Journal of Social Entrepreneurship, 6}(3), 352-378. doi:10.1080/19420676.2015.1049285

Ormiston, J., \& Seymour, R. (2011). Understanding value creation in social entrepreneurship: The importance of aligning mission, strategy and impact measurement. \emph{Journal of Social Entrepreneurship, 2}(2), 125-150. doi:10.1080/19420676.2011.606331

Pampel, F. C. (2000). \emph{Logistic regression: A primer}. Sage.

Parker, S. C. (2006). \emph{Entrepreneurship, self-employment and the labour market}. In Basu, A., Casson, M., Wadeson, N., \& Yeung, B. (Eds.). \emph{The} \emph{Oxford handbook of entrepreneurship. }(pp.435-460). Springer. 

Parker, S. C. (2009). \emph{The economics of entrepreneurship}. Cambridge: Cambridge University Press.

Pathak, S., \& Muralidharan, E. (2016). Informal institutions and their comparative influences on social and commercial entrepreneurship: The role of in-group collectivism and interpersonal Trust. \emph{Journal of Small Business Management, 54}(S1), 168-188. doi:10.1111/jsbm.12289

Pathak, S., \& Muralidharan, E. (2018). Economic inequality and social entrepreneurship. \emph{Business \& Society, 57}(6), 1150-1190. doi:10.1177/0007650317696069

Peredo, A. M., \& McLean, M. (2006). Social entrepreneurship: A critical review of the concept. \emph{Journal of World Business, 41}(1), 56-65. doi:10.1016/j.jwb.2005.10.007

Phillips, S. D., \& Johnson, B. (2019). Inching to Impact: The Demand Side of Social Impact Investing. \emph{Journal of Business Ethics}, \emph{168}, 615-629. doi:10.1007/s10551-019-04241-5

Pierson, P. (1996). The new politics of the welfare state. \emph{World Politics, 48}(2), 143-179. doi:10.1353/wp.1996.0004

Piff, P. K., Stancato, D. M., Martinez, A. G., Kraus, M. W., \& Keltner, D. (2012). Class, chaos, and the construction of community. \emph{Journal of Personality and Social Psychology, 103}(6), 949-962. doi:10.1037/a0029673

Priya, S. S., Cuce, E., \& Sudhakar, K. (2021). A perspective of COVID 19 impact on global economy, energy and environment. \emph{International Journal of Sustainable Engineering, 14}(6), 1290-1305. doi:10.1080/19397038.2021.1964634

Rahdari, A., Sepasi, S., \& Moradi, M. (2016). Achieving sustainability through Schumpeterian social entrepreneurship: The role of social enterprises. \emph{Journal of Cleaner Production, 137}, 347-360. doi:10.1016/j.jclepro.2016.06.159

Ramus, T., \& Vaccaro, A. (2017). Stakeholders matter: How social enterprises address mission drift. \emph{Journal of Business Ethics, 143}(2), 307-322. doi:10.1007/s10551-014-2353-y

Rapp, C., Shore, J., \& Tosun, J. (2018). Not so risky business? How social policies shape the perceived feasibility of self-employment. \emph{Journal of European Social Policy, 28}(2), 143-160. doi:10.1177/0958928717711973

Rawhouser, H., Cummings, M., \& Newbert, S. L. (2019). Social impact measurement: Current approaches and future directions for social entrepreneurship research. \emph{Entrepreneurship Theory and Practice, 43}(1), 82-115. doi:10.1177/1042258717727718

Reeskens, T., \& van Oorschot, W. (2014). European feelings of deprivation amidst the financial crisis: Effects of welfare state effort and informal social relations. \emph{Acta }\emph{Sociologica}\emph{, 57}(3), 191-206. doi:10.1177/0001699313504231

Rey-Martí, A., Ribeiro-Soriano, D., \& Sánchez-García, J. L. (2016). Giving back to society: Job creation through social entrepreneurship. \emph{Journal of Business Research, 69}(6), 2067-2072. doi:10.1016/j.jbusres.2015.12.010

Reynolds, P., Bosma, N., Autio, E., Hunt, S., De Bono, N., Servais, I., . . . Chin, N. (2005). Global entrepreneurship monitor: Data collection design and implementation 1998--2003. \emph{Small Business Economics, 24}(3), 205-231. doi:10.1007/s11187-005-1980-1

Roller, E. (1995). \emph{The welfare state: The equality dimension}. New York/Oxford: Oxford University Press.

Roosma, F., Gelissen, J., \& Van Oorschot, W. (2013). The multidimensionality of welfare state attitudes: A European cross-national study. \emph{Social Indicators Research, 113}(1), 235-255. doi:10.1007/s11205-012-0099-4

Roosma, F., \& Jeene, M. (2017). The deservingness logic applied to public opinions concerning work obligations for benefit claimants. In Van Oorschot, W., Roosma, F., Meuleman, B., \& Reeskens, T. (Eds.). \emph{The Social Legitimacy of Targeted Welfare: Attitudes to welfare deservingness} (pp. 189 -- 206). Cheltenham: Edward Elgar Publishing.

Roosma, F., Van Oorschot, W., \& Gelissen, J. (2016). The Achilles' heel of welfare state legitimacy: perceptions of overuse and underuse of social benefits in Europe. \emph{Journal of European Public Policy, 23}(2), 177-196. doi:10.1080/13501763.2015.1031157

Rose, D., \& Harrison, E. (2007). The European socio-economic classification: a new social class schema for comparative European research. \emph{European Societies, 9}(3), 459-490. doi:10.1080/14616690701336518

Rothstein, B. (2001). Social capital in the social democratic welfare state. \emph{Politics \& Society, 29}(2), 207-241. doi:10.1177/0032329201029002003

Roy, M. J., Donaldson, C., Baker, R., \& Kay, A. (2013). Social enterprise: new pathways to health and well-being? \emph{Journal of Public Health }\emph{Holicy}\emph{, 34}(1), 55-68. doi:10.1057/jphp.2012.61

Saebi, T., Foss, N. J., \& Linder, S. (2019). Social entrepreneurship research: Past achievements and future promises. \emph{Journal of Management, 45}(1), 70-95. doi:10.1177/0149206318793196

Salamon, L. M. (2002). The tools of government. A guide to the New Governance. \emph{New York: Oxford University Press}. 

Salamon, L. M., \& Anheier, H. K. (1998). Social origins of civil society: Explaining the nonprofit sector cross-nationally. \emph{VOLUNTAS: International Journal of Voluntary and }\emph{Nonprofit}\emph{ Organizations, 9}(3), 213-248. 

Salamon, L. M., \& Sokolowski, S. W. (2003). Institutional roots of volunteering: Toward a macro-structural theory of individual voluntary action. In Dekker, P., \& Halman, L. (Eds.). \emph{The values of volunteering}. (pp. 71-90). Springer.

Salamon, L. M., Sokolowski, S. W., \& Anheier, H. K. (2000). Social origins of civil society: An overview. \emph{Working Papers of the Johns Hopkins Comparative }\emph{Nonprofit}\emph{ Sector Project}, no. 38. Baltimore: The Johns Hopkins Center for Civil Society Studies.

Salamon, L. M., \& Toepler, S. (2015). Government--nonprofit cooperation: Anomaly or necessity? \emph{VOLUNTAS: International Journal of Voluntary and }\emph{Nonprofit}\emph{ Organizations, 26}(6), 2155-2177. doi:10.1007/s11266-015-9651-6

Salazar, J., Husted, B. W., \& Biehl, M. (2012). Thoughts on the evaluation of corporate social performance through projects. \emph{Journal of Business Ethics, 105}(2), 175-186. doi:10.1007/s10551-011-0957-z

Sandfort, J., Selden, S. C., \& Sowa, J. E. (2008). Do government tools influence organizational performance? Examining their implementation in early childhood education. \emph{The American Review of Public Administration, 38}(4), 412-438. doi:10.1177/0275074007310488

Santos, F. M. (2012). A positive theory of social entrepreneurship. \emph{Journal of Business Ethics, 111}(3), 335-351. doi:10.1007/s10551-012-1413-4

Santos, H. C., Varnum, M. E., \& Grossmann, I. (2017). Global increases in individualism. \emph{Psychological Science, 28}(9), 1228-1239. doi:10.1177/0956797617700622

Sarracino, F., \& Fumarco, L. (2018). Assessing the non-financial outcomes of social enterprises in Luxembourg. \emph{Journal of Business Ethics}, 1-27. doi:10.1007/s10551-018-4086-9

Scherer, A. G., \& Palazzo, G. (2011). The new political role of business in a globalized world: A review of a new perspective on CSR and its implications for the firm, governance, and democracy. \emph{Journal of Management Studies, 48}(4), 899-931. doi:10.1111/j.1467-6486.2010.00950.x

Schjoedt, L. (2009). Entrepreneurial job characteristics: An examination of their effect on entrepreneurial satisfaction. \emph{Entrepreneurship Theory and Practice, 33}(3), 619-644. doi:10.1111/j.1540-6520.2009.00319.x

Schlaegel, C., \& Koenig, M. (2014). Determinants of entrepreneurial intent: A meta--analytic test and integration of competing models. \emph{Entrepreneurship Theory and Practice, 38}(2), 291-332. doi:10.1111/etap.12087

Schumpeter, J. A. (1934). \emph{The theory of economic development. }Cambridge, MA: Harvard University Press.

Schumpeter, J. A. (1942). \emph{Capitalism, socialism and democracy}. New York Harper \& Bros.

Schwab, K., \& Porter, M. (2008). \emph{The global competitiveness report 2008--2009} (9295044118). Retrieved from https://www3.weforum.org/docs/WEF\_GlobalCompetitivenessReport\_2008-09.pdf

Scott, W. R. (2005). Institutional theory: Contributing to a theoretical research program. In Smith, K. G., Hitt, M. A. (Eds.). \emph{Great minds in management: The process of theory development}, (pp.460-484). Oxford: Oxford University Press. 

Scott, W. R. (2013). \emph{Institutions and organizations: Ideas, interests, and identities}. London: Sage publications.

Scruggs, L., \& Allan, J. (2006). Welfare-state decommodification in 18 OECD countries: a replication and revision. \emph{Journal of European Social Policy, 16}(1), 55-72. doi:10.1177/0958928706059833

Seelos, C., \& Mair, J. (2005). Social entrepreneurship: Creating new business models to serve the poor. \emph{Business Horizons, 48}(3), 241-246. doi:10.1016/j.bushor.2004.11.006

Segal, G., Borgia, D., \& Schoenfeld, J. (2005). The motivation to become an entrepreneur. \emph{International Journal of Entrepreneurial }\emph{Behavior}\emph{ \& Research}, \emph{11}(1), 42-57. doi:10.1108/13552550510580834

Short, J. C., Moss, T. W., \& Lumpkin, G. T. (2009). Research in social entrepreneurship: past contributions and future opportunities. \emph{Strategic Entrepreneurship Journal, 3}(2), 161-194. doi:10.1002/sej.69

Smith, B. R., \& Stevens, C. E. (2010). Different types of social entrepreneurship: The role of geography and embeddedness on the measurement and scaling of social value. \emph{Entrepreneurship and regional development, 22}(6), 575-598. doi:10.1080/08985626.2010.488405

Snijders, T. A. B., \& Bosker, R. J. (2012). \emph{Multilevel analysis: An introduction to basic and advanced multilevel }\emph{modeling}. Sage Publications.

Solomon, S., Bendickson, J. S., Liguori, E. W., \& Marvel, M. R. (2021). The effects of social spending on entrepreneurship in developed nations. \emph{Small Business Economics}, \emph{58}, 1595-1607. doi:10.1007/s11187-021-00458-9

Sommet, N., \& Morselli, D. (2017). Keep calm and learn multilevel logistic modeling: A simplified three-step procedure using Stata, R, Mplus, and SPSS. \emph{International Review of Social Psychology, 30}(1). doi:10.5334/irsp.90

Spear, R., \& Bidet, E. (2005). Social enterprise for work integration in 12 European countries: a descriptive analysis. \emph{Annals of public and cooperative economics, 76}(2), 195-231. doi:10.1111/j.1370-4788.2005.00276.x

Stadelmann-Steffen, I. (2011). Social volunteering in welfare states: Where crowding out should occur. \emph{Political Studies, 59}(1), 135-155. doi:10.1111/j.1467-9248.2010.00838.x

Steenbergen, M. R., \& Jones, B. S. (2002). Modeling multilevel data structures. \emph{American Journal of Political Science}, \emph{46}(1), 218-237. doi:10.2307/3088424 

Stegmueller, D. (2013). How many countries for multilevel modeling? A comparison of frequentist and Bayesian approaches. \emph{American Journal of Political Science, 57}(3), 748-761. doi:10.1111/ajps.12001

Stephan, U., \& Drencheva, A. (2017). \emph{The person in social entrepreneurship: A systematic review of research on the social entrepreneurial personality}. In Ahmetoglu, G., Chamorro-Premuzic, T., Klinger, B., \& Karcisky, T. (Eds.). \emph{The Wiley handbook of entrepreneurship}. (pp205-229). Chichester: Wiley. 

Stephan, U., \& Folmer, E. (2017). Context and social enterprises: which environments enable social entrepreneurship? \emph{European Policy Brief. }Retrieved from https://publications.aston.ac.uk/id/eprint/31317/1/SEFORIS\_POLICY\_BRIEF\_WP7\_context\_and\_social\_enterprise\_part\_1.pdf

Stephan, U., Patterson, M., Kelly, C., \& Mair, J. (2016). Organizations driving positive social change: A review and an integrative framework of change processes. \emph{Journal of Management, 42}(5), 1250-1281. doi:10.1177/0149206316633268

Stephan, U., Uhlaner, L. M., \& Stride, C. (2015). Institutions and social entrepreneurship: The role of institutional voids, institutional support, and institutional configurations. \emph{Journal of International Business Studies, 46}(3), 308-331. doi:doi:10.1057/jibs.2014.38

Stevens, R., Moray, N., \& Bruneel, J. (2014). The Social and Economic Mission of Social Enterprises: Dimensions, Measurement, Validation, and Relation. \emph{Entrepreneurship Theory and Practice, 39}(5), 1-32. doi:10.1111/etap.12091

Stirzaker, R., Galloway, L., Muhonen, J., \& Christopoulos, D. (2021). The drivers of social entrepreneurship: agency, context, compassion and opportunism. \emph{International Journal of Entrepreneurial }\emph{Behavior}\emph{ \& Research}. \emph{27}(6), 1381-1402. doi:10.1108/IJEBR-07-2020-0461

Sud, M., VanSandt, C. V., \& Baugous, A. M. (2009). Social entrepreneurship: The role of institutions. \emph{Journal of Business Ethics, 85}(1), 201-216. doi:10.1007/s10551-008-9939-1

Sunley, P., \& Pinch, S. (2012). Financing social enterprise: social bricolage or evolutionary entrepreneurialism? \emph{Social Enterprise Journal}, \emph{8}(2), 108-122. doi:10.1108/17508611211252837

Tan, L. P., Le, A. N. H., \& Xuan, L. P. (2020). A systematic literature review on social entrepreneurial intention. \emph{Journal of Social Entrepreneurship, 11}(3), 241-256. doi:10.1080/19420676.2019.1640770

Terjesen, S., Bosma, N., \& Stam, E. (2016). Advancing public policy for high-growth, female, and social entrepreneurs. \emph{Public Administration Review, 76}(2), 230-239. doi:10.1111/puar.12472

Thompson, J., Alvy, G., \& Lees, A. (2000). Social entrepreneurship--a new look at the people and the potential. \emph{Management decision, 38}(5), 328-338. doi:10.1108/00251740010340517

Thompson, N., Kiefer, K., \& York, J. G. (2011). \emph{Distinctions not dichotomies: Exploring social, sustainable, and environmental entrepreneurship}. In Lumpkin, G. T., \& Katz, J. A. (Eds.). \emph{Social and sustainable entrepreneurship (Advances in entrepreneurship, firm emergence and growth)}. (pp.201-229). Emerald Group Publishing Limited.

Townsend, D. M., \& Hart, T. A. (2008). Perceived institutional ambiguity and the choice of organizational form in social entrepreneurial ventures. \emph{Entrepreneurship Theory and Practice, 32}(4), 685-700. doi:10.1111/j.1540-6520.2008.00248.x

Uzzi, B. (1997). Social structure and competition in interfirm networks: The paradox of embeddedness. \emph{Administrative}\emph{ }\emph{Science}\emph{ }\emph{Quarterly}, \emph{42}(1), 35-67. doi:10.2307/2393808

Van Den Groenendaal, S. M. E., Rossetti, S., Van Den Bergh, M., Kooij, T. D., \& Poell, R. F. (2021). Motivational profiles and proactive career behaviors among the solo self-employed. \emph{Career}\emph{ Development International, 26}(2), 309-330. doi:10.1108/CDI-06-2020-0149

Van der Wel, K. A., \& Halvorsen, K. (2015). The bigger the worse? A comparative study of the welfare state and employment commitment. \emph{Work, Employment and Society, 29}(1), 99-118. doi:10.1177/0950017014542499

Van Oorschot, W. (2010). Public perceptions of the economic, moral, social and migration consequences of the welfare state: An empirical analysis of welfare state legitimacy. \emph{Journal of European Social Policy, 20}(1), 19-31. doi:10.1177/0958928709352538

Van Oorschot, W., \& Arts, W. (2005). The social capital of European welfare states: the crowding out hypothesis revisited. \emph{Journal of European Social Policy, 15}(1), 5-26. doi:10.1177/0958928705049159 

Van Oorschot, W., Arts, W., \& Halman, L. (2005). Welfare state effects on social capital and informal solidarity in the European Union: evidence from the 1999/2000 European Values Study. \emph{Policy \& Politics, 33}(1), 33-54. doi:10.1332/0305573052708474

Van Oorschot, W., Reeskens, T., \& Meuleman, B. (2012). Popular perceptions of welfare state consequences: A multilevel, cross-national analysis of 25 European countries. \emph{Journal of European Social Policy, 22}(2), 181-197. doi:10.1177/0958928711433653

van Oorschot, W., Roosma, F., Meuleman, B., \& Reeskens, T. (2017). \emph{The social legitimacy of targeted welfare: Attitudes to welfare deservingness}: Edward Elgar Publishing.

van Rijn, M., Raab, J., Roosma, F., \& Achterberg, P. (2021). To Prove and Improve: An Empirical Study on Why Social Entrepreneurs Measure Their Social Impact. \emph{Journal of Social Entrepreneurship}, 1-23. doi:10.1080/19420676.2021.1975797

Verheul, I., Wennekers, S., Audretsch, D., \& Thurik, R. (2002). \emph{An eclectic theory of entrepreneurship: policies, institutions and culture}. In Audretsch, D., Thurik, R., Verheul, I., \& Wennekers, S. (Eds.). \emph{Entrepreneurship: Determinants and policy in a European-US comparison} (pp.11-81). Kluwer Academic Publishers.

Vervoort, M. (2012). Ethnic concentration in the neighbourhood and ethnic minorities' social integration: Weak and strong social ties examined. \emph{Urban Studies, 49}(4), 897-915. doi:10.1177/0042098011408141

Visser, M., Gesthuizen, M., \& Scheepers, P. (2018). The crowding in hypothesis revisited: new insights into the impact of social protection expenditure on informal social capital. \emph{European Societies, 20}(2), 257-280. doi:10.1080/14616696.2018.1442928

Wang, Z., Jetten, J., \& Steffens, N. K. (2019). The more you have, the more you want? Higher social class predicts a greater desire for wealth and status. \emph{European Journal of Social Psychology, 50}(2), 360-375. doi:10.1002/ejsp.2620

Warner, R. M. (2012). \emph{Applied statistics: From bivariate through multivariate techniques}. Sage Publications.

Warr, P. G. (1982). Pareto optimal redistribution and private charity. \emph{Journal of Public Economics, 19}(1), 131-138. doi:10.1016/0047-2727(82)90056-1

Weerawardena, J., \& Mort, G. S. (2006). Investigating social entrepreneurship: A multidimensional model. \emph{Journal of World Business, 41}(1), 21-35. doi:10.1016/j.jwb.2005.09.001

Welzel, C. (2013). \emph{Freedom rising: Human empowerment and the quest for emancipation}. New York, NY: Cambridge University Press.

Wendling, Z. A., Emerson, J. W., Esty, D. C., Levy, M. A., De Sherbinin, A., \& Emerson, J. (2018). Environmental performance index. \emph{Yale }\emph{Center}\emph{ for Environmental Law \& Policy: New Haven, CT, USA}. 

Wennekers, S., Van Wennekers, A., Thurik, R., \& Reynolds, P. (2005). Nascent entrepreneurship and the level of economic development. \emph{Small Business Economics, 24}(3), 293-309. doi:10.1007/s11187-005-1994-8

Wilkinson, C., Medhurst, J., Henry, N., Wihlborg, M., \& Braithwaite, B. W. (2015). \emph{A map of social enterprises and their eco-systems in Europe: Synthesis Report}. Retrieved from https://ec.europa.eu/social/BlobServlet?docId=12987\&langId=en

WVS. (2014). \emph{World Values Survey: Round Five - Country-Pooled Datafile Version: www.worldvaluessurvey.org/WVSDocumentationWV5.jsp. }\emph{Madrid: JD Systems Institute.} 

Yitshaki, R., \& Kropp, F. (2016). Motivations and opportunity recognition of social entrepreneurs. \emph{Journal of Small Business Management, 54}(2), 546-565. doi:10.1111/jsbm.12157

Young, D. R. (2000). Alternative models of government-nonprofit sector relations: Theoretical and international perspectives. \emph{Nonprofit}\emph{ and Voluntary Sector Quarterly, 29}(1), 149-172. doi:10.1177/0899764000291009

Young, D. R. (2008). \emph{A unified theory of social enterprise}. In Shockley, G. E., Frank, P. M., \& Stough, R. R. (Eds.). \emph{Non-market entrepreneurship. }(pp.175-191). Edward Elgar Publishing Limited. 

Zahra, S., Gedajlovic, E., Neubaum, D., \& Shulman, J. (2009). A typology of social entrepreneurs: Motives, search processes and ethical challenges. \emph{Journal of Business Venturing, 24}(5), 519-532. doi:10.1016/j.jbusvent.2008.04.007

Zahra, S., Newey, L., \& Li, Y. (2014). On the frontiers: The implications of social entrepreneurship for international entrepreneurship. \emph{Entrepreneurship Theory and Practice, 38}(1), 137-158. doi:10.1111/etap.12061

Zahra, S., Rawhouser, H. N., Bhawe, N., Neubaum, D. O., \& Hayton, J. C. (2008). Globalization of social entrepreneurship opportunities. \emph{Strategic Entrepreneurship Journal, 2}(2), 117-131. doi:10.1002/sej.43

Zahra, S., \& Wright, M. (2011). Entrepreneurship's next act. \emph{The Academy of Management Perspectives, 25}(4), 67-83. doi:10.5465/amp.2010.0149





\end{document}
